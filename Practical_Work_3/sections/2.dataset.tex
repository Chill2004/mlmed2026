\section{Dataset}
We use the COVID-QU-Ex dataset for infection segmentation. Each sample contains a grayscale chest X-ray image and a corresponding infection mask. The mask indicates the infection region using pixel-level labels.

\subsection{Subset Choice (COVID-only)}
The full dataset contains several categories (COVID-19, Non-COVID, Normal) and standard splits (Train/Val/Test). Due to the large size of the full dataset and limited computing resources, we choose to train and evaluate using only the \textbf{COVID-19 subset}. 

\subsection{Preprocessing}
All images and masks are resized to $256\times256$. Images are normalized to the range $[0,1]$. Masks are binarized using a threshold of 0.5 (infection vs background).

\subsection{Visual Check}
We first visualize random samples to confirm that images and masks are aligned correctly. Figure~\ref{fig:overlay_check} shows an example of a chest X-ray with the infection mask overlay.

\begin{figure}[H]
\centering
\includegraphics[width=0.95\linewidth]{figure/example.png}
\caption{Example overlay of infection mask on a COVID-19 chest X-ray.}
\label{fig:overlay_check}
\end{figure}
