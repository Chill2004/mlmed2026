\section{Methods}
\subsection{Problem formulation}
Given an ultrasound image $I$ and its pixel size $s$ (mm/pixel), the goal is to predict head circumference $\hat{y}$ in millimeters.
We train using mean absolute error (MAE), equivalent to L1 loss:
\[
\mathcal{L} = \frac{1}{N}\sum_{i=1}^{N} \left|\hat{y}_i - y_i\right|.
\]

\subsection{Preprocessing}
Ultrasound frames often include black borders and non-informative areas outside the scanning region.
We apply:
\begin{itemize}
  \item \textbf{Auto-cropping:} remove near-black borders by cropping the non-black region
  \item \textbf{Resizing:} resize to a fixed input resolution (e.g., $256\times256$)
  \item \textbf{Normalization:} scale intensities to $[0,1]$
\end{itemize}

\subsection{Model architecture (ResNet-18 regression)}
We use ResNet-18 as the main backbone and adjust the first convolution layer so it can take one-channel (grayscale) ultrasound images.
After the backbone, global average pooling is used to extract a compact feature vector that summarizes the image.
Because HC is measured in millimeters and depends on image scale, we also include the pixel size $s$ as additional input.
We concatenate $s$ with the image feature vector and feed the combined representation into a small MLP regression head to predict HC in mm.


\subsection{Training details}
We split the training set into train/validation subsets using an 80/20 split with a fixed random seed.
We train using Adam and monitor validation MAE.

\begin{itemize}
  \item Input size: \textbf 256
  \item Batch size: \textbf 32
  \item Learning rate: \textbf 0.0003
  \item Weight decay: \textbf 0.0001
  \item Epochs: \textbf 20
  \item Augmentation (optional): random flip and small rotations
\end{itemize}

\subsection{Evaluation protocol}
We report MAE on the internal validation split.
We plot both training MAE and validation MAE across epochs to diagnose underfitting/overfitting.
