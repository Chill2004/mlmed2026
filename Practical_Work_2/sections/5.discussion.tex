\section{Discussion}
From our experiments, we find that accuracy depends a lot on image quality and how clearly the skull boundary can be seen.
Some images contain strong speckle noise, low contrast around the skull edge, or only part of the head, which makes prediction more difficult.
In addition, different acquisition settings and zoom levels can change the scale of the image.

One important design choice in our model is to include the pixel size (mm/pixel) as an input feature.
If pixel size is not provided, the model must guess the real-world scale only from the image appearance, and this can be unreliable when zoom changes.
By concatenating pixel size with CNN features, the model can better predict HC in real units (mm).

A limitation of our regression approach is that it is less interpretable than segmentation-based pipelines.
With segmentation and ellipse fitting, we can visualize the predicted boundary, which is easier to check and may generalize better in some situations.
In the future, performance could be improved by using stronger but anatomically safe augmentations, applying more robust cropping of the ultrasound fan region, and using multi-task learning to predict both HC and a coarse boundary representation.
